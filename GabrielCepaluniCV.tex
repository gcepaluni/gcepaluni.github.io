% LaTeX Curriculum Vitae Template
%
% Copyright (C) 2004-2009 Jason Blevins <jrblevin@sdf.lonestar.org>
% http://jblevins.org/projects/cv-template/
%
% You may use use this document as a template to create your own CV
% and you may redistribute the source code freely. No attribution is
% required in any resulting documents. I do ask that you please leave
% this notice and the above URL in the source code if you choose to
% redistribute this file.



\documentclass[a4paper,11.5pt]{article}
\usepackage[T1]{fontenc}
\usepackage{libertine}
\usepackage[libertine]{newtxmath}
\usepackage{inconsolata}
\usepackage{hyperref}
\usepackage[left=1.0in,top=2.5cm,bottom=2.5cm]{geometry}
\usepackage[usenames,dvipsnames]{xcolor}
\usepackage{eurosym} % euro sign
\definecolor{darkblue}{rgb}{0.0, 0.0, 0.55}
\usepackage[english,brazilian]{babel}
\usepackage[UKenglish]{isodate}
\exhyphenpenalty=1000
\hyphenpenalty=1000
\widowpenalty=1000
\clubpenalty=1000



% Set your name here
\def\name{Gabriel Cepaluni}

% Replace this with a link to your CV if you like, or set it empty
% (as in \def\footerlink{}) to remove the link in the footer:

%\def\footerlink{http://danilofreire.github.io/DaniloFreireCV.pdf}

% The following metadata will show up in the PDF properties
\hypersetup{
	colorlinks = true,
	urlcolor = darkblue,
	pdfauthor = {Gabriel Cepaluni},
	pdfkeywords = {},
	pdftitle = {Gabriel Cepaluni: Curriculum Vitae},
	pdfsubject = {},
	pdfpagemode = UseNone
}

% Customise page headers
\pagestyle{myheadings}
\markright{Gabriel Cepaluni}
\thispagestyle{empty}

% Custom section fonts
\usepackage{sectsty}
\sectionfont{\rmfamily\mdseries\Large}
\subsectionfont{\rmfamily\mdseries\itshape\large}

% Other possible font commands include:
% \ttfamily for teletype,
% \sffamily for sans serif,
% \bfseries for bold,
% \scshape for small caps,
% \normalsize, \large, \Large, \LARGE sizes.

% Don't indent paragraphs.
\setlength\parindent{0em}

% Make lists without bullets
\renewenvironment{itemize}{
	\begin{list}{}{
			\setlength{\leftmargin}{1.5em}
		}
		}{
	\end{list}
}


\begin{document}
\sloppy

% Place name at left
{\huge \bf \name}

% Alternatively, print name centred and bold:
%\centerline{\huge \bf \name}

\vspace{0.25in}

\begin{minipage}{0.45\linewidth}
	\textbf{São Paulo State University}\\
	School of Humanities and Social Sciences\\
  Av. Eufrásia Monteiro Petráglia, 900, \\
	Jardim Dr. Antonio Petraglia, \\
	Franca - SP, Brazil, 14409-160
\end{minipage}
\begin{minipage}{0.45\linewidth}
	\begin{tabular}{ll}
	Citizenship:   &   Brazilian and Italian          \\
		Email:          & \href{mailto:gabriel.cepaluni@unesp.br}{gabriel.cepaluni@unesp.br}            \\
	%	Homepage:       & \href{http://danilofreire.github.io/}{http://danilofreire.github.io}   \\
    Google Scholar: & \href{https://tinyurl.com/2p995swk}{https://tinyurl.com/2p995swk} %\\
	%	GitHub:         & \href{http://github.com/danilofreire}{http://github.com/danilofreire}
	\end{tabular}
	\end{minipage}

\section*{Academic Appointments}

\begin{itemize}
	\item 2017 -- pres. Associate Professor (with tenure), Department of International Relations and Public Policy, São Paulo State University.
	\item 2010 -- 2017. Assistant Professor, Department of International Relations, São Paulo State University.
\end{itemize}

\section*{Education}

\begin{itemize}
	\item 2017. \href{https://en.wikipedia.org/wiki/Habilitation}{Habilitation (or \emph{Livre-Docência}) in International Relations,} University of São Paulo.
	\item 2006 -- 2010. Ph.D. in Political Science, University of São Paulo.
	\item 2003 -- 2004. MA in International Relations, San Tiago Dantas Program – UNESP, UNICAMP and PUC-SP.
	\item 1996 -- 2002. BA in Social Sciences, University of São Paulo.
	\item 1996 -- 1999. BA in Journalism, Methodist University of São Paulo.
\end{itemize}

\section*{Visiting fellowships}

\begin{itemize}
	
	\item 2019 -- 2020. Senior Global Challenges Fellow at Institute for Advanced Study at Central European University (IAS CEU).
	
	\item 2017. Visiting Researcher at \emph{Universidad Nacional del Sur} (UNS) / \emph{Consejo Nacional de Investigaciones Científicas y Técnicas} (CONICET).
	
	\item 2016. Visiting Fellow at European Union Visitors Programme (EUVP), European Union, Brussels, Belgium.
	
	\item 2016. Visiting Researcher at \emph{Fondation Maison des Sciences de l’Homme/ DEA - Directeurs d'Études Associés}, Paris, France.
	
	\item 2016. Guest Researcher at Uppsala Forum on Democracy, Peace, and Justice, Sweden.
	
	\item 2014 -- 2015. Visiting Researcher at Department of Politics, NYU.
	
	\item 2014. Visiting Researcher at Watson Institute for International Studies, Brown University.
	
	\item 2013. Visiting Lecturer at the \emph{La Salle University}, Bogotá, Colombia.
	
	\item  2012 -- 2013. Guest Researcher at the Department of Political Science at Aarhus University, Denmark.
	
	\item  2008 -- 2009. Visiting Researcher at the Department of Government at Georgetown University.
	
\end{itemize}

\section*{Selected Publications}

\section*{Book}

\begin{itemize}
	\item 2010. Vigevani, Tullo, and \textbf{Cepaluni, Gabriel.} \href{https://www.google.com.br/books/edition/Brazilian_Foreign_Policy_in_Changing_Tim/A2AmOQpQPYcC?hl=pt-BR&gbpv=1&dq=Brazilian+Foreign+Policy&printsec=frontcover}{Brazilian Foreign Policy in Changing Times: The Quest for Autonomy from Sarney to Lula.} MD: Lexington Books. (Foreword by Prof. Philippe C. Schmitter). Portuguese edition from \emph{Editora da Unesp} (São Paulo), 1st edition, 2012; 2nd edition, 2016. Chinese edition from Social Sciences Academic Press (Beijing), 2015. [Google Scholar: over 140 citations in Portuguese; Google Scholar: over 180 citations in English.]
\end{itemize}

\subsection*{Peer-Reviewed Articles}

\begin{itemize}
	
	\item \emph{Forthcoming.} Mignozzetti, Umberto, \textbf{Cepaluni, Gabriel}, and Freire, Danilo. \href{https://github.com/umbertomig/legislature-size-welfare}{Legislature Size and Welfare: Evidence From Brazil}. \textit{American Journal of Political Science}. [Journal 5-Year Impact Factor: 9.534]
	
	\item 2022. \textbf{Cepaluni, Gabriel}, Dorsch, Michael T., and Kovarek, Daniel. \href{https://internal-journal.ssph-journal.org/articles/10.3389/ijph.2022.1604663/full}{Mobility and Policy Responses During the COVID-19 Pandemic.} \textit{International Journal of Public Health}, 67(1604663), p. 1-8. [Journal 5-Year Impact Factor: 4.203]
	
	\item 2022. \textbf{Cepaluni, Gabriel}, Chewning, Taylor K., Driscoll, Amanda, and  Faganello, Marco A. \href{https://www.sciencedirect.com/science/article/pii/S0305750X21003831?casa_token=seI6YOJMdUAAAAAA:6-vwCPsSxext792LA9tHLHo8Bex1Q2MdfmAxZOF7GNJGYZ8oBRM-Laldb48bHRa-zdiP29DpQw}{Conditional Cash Transfers and Child Labor.} \textit{World Development}, 152 (105768), p. 1-15. [Journal 5-Year Impact Factor: 7.324]
	
	\item 2021. \textbf{Cepaluni, Gabriel}, Dorsch, Michael T., and Branyiczki, Réka. \href{https://bristoluniversitypressdigital.com/view/journals/jpfpc/37/1/article-p27.xml}{Political Regimes and Deaths in the Early Stages of the COVID-19 Pandemic.} \textit{Journal of Public Finance and Public Choice}, 37 (1), p. 27-53. [Google Scholar: around 100 citations.]
	
	\item 2020. \textbf{Cepaluni, Gabriel}, and Fernandes, I. F. \href{https://tinyurl.com/yw8px7h9}{United We Stand, Divided We Fall: Coalitions in the GATT/WTO negotiations,} \textit{International Political Science Review}, 43 (4), p. 1-17. [Journal 5-Year Impact Factor: 2.648]
	
	\item 2018. Driscoll, Amanda, \textbf{Cepaluni, Gabriel}, Guimarães, Feliciano D. S., and Spada, Paolo. \href{https://tinyurl.com/4xw2vtct}{Prejudice, Strategic Discrimination, and the Electoral Connection: Evidence from a Pair of Field Experiments in Brazil.} \textit{American Journal of Political Science}, 62 (4), p. 781-795 (lead article). [Journal 5-Year Impact Factor: 9.534]
	
	\item 2016. \textbf{Cepaluni, Gabriel}, and Hidalgo, F. Daniel. \href{https://www.cambridge.org/core/journals/political-analysis/article/abs/compulsory-voting-can-increase-political-inequality-evidence-from-brazil/76BB8B11EA9A3FF75A1B0591C663B303}{Compulsory Voting Can Increase Political Inequality: Evidence from Brazil.} \textit{Political Analysis}, 24 (2), p. 273-280.  [Journal 5-Year Impact Factor: 8.310] [Google Scholar: around 100 citations.]
	
	\item 2007. Vigevani, Tullo, and \textbf{Cepaluni, Gabriel.} \href{https://tinyurl.com/y5wk2js5}{Lula's Foreign Policy and the Quest for Autonomy through Diversification.}  \textit{Third World Quarterly}, 28 (7), p. 1309-1326. (An expanded version appeared in Portuguese in \emph{Contexto Internacional}, 2007.)
	[Journal 5-Year Impact Factor: 3.007] [Google Scholar: over 840 citations in Portuguese; Google Scholar: over 185 citations in English.]
\end{itemize}

\subsection*{Book Chapters}

\begin{itemize}
	\item 2019. \textbf{Cepaluni, Gabriel,} Mariano, Karina L. Pasquariello, and Mariano, Marcelo Passini.  \href{https://link.springer.com/chapter/10.1007/978-3-319-99552-6_4}{``Preserving Domestic Autonomy: Weak Migration Laws and the Mercosur Strategy of Limited Integration"}, In: César Álvarez Alonso; José Ignacio Hernández. (eds.). \textit{Latin American Geopolitics}, London, UK: Palgrave Macmillan, p. 83--107.
	
	\item 2013. Vigevani, Tullo, and \textbf{Cepaluni, Gabriel.} \href{https://tinyurl.com/4k8c5d2m}{``Brazil: Global Power-To-Be?".} In: Ryan K. Beasley; Juliet Kaarbo; Jeffrey S. Lantis; Michael T. Snarr. (eds.). Foreign Policy in Comparative Perspective: Domestic and International Influences on State Behavior. 1ed. Washington, DC: CQ Press, p. 265-290. [Google Scholar: the book has over 200 citations.]
\end{itemize}

\subsection*{Working Papers}

\begin{itemize}
	
	\item 2022. \textbf{Cepaluni, Gabriel}, Dorsch, Michael T., and Dzebo, Semir. \href{https://papers.ssrn.com/sol3/papers.cfm?abstract_id=3816398}{Populism, Political Regimes, and COVID-19 Deaths.} revision submitted.
	
	\item 2022. \textbf{Cepaluni, Gabriel}, Dorsch, Michael T., and Civitarese, Jamil. Land Invasions and Contemporary Slavery.
	
	\item 2022. \textbf{Cepaluni, Gabriel}, and Driscoll, Amanda. Can Conditional Cash Transfer Interrupt the Cycle of Intergenerational Poverty? Lessons from a Large Administrative Data Set.
	
\end{itemize}

\subsection*{Work in Progress}

\begin{itemize}
	\item 2022. \textbf{Cepaluni, Gabriel}, Dorsch, Michael T., Kovarek, Daniel, and Moise, Alexandru. Protests and Representation: BLM Protests and Voting for Minorities in U.S. Congressional Elections.
	\item 2022. \textbf{Cepaluni, Gabriel}, Civitarese, Jamil, Freire, Danilo, and Mignozzetti, Umberto. Career paths of bureaucrats and politicians in Brazil.
	\item 2022. Baragwanath, Kathryn, \textbf{Cepaluni, Gabriel}, and Mignozzetti, Umberto. Droughts and Votes: Evidence from São Paulo.
\end{itemize}

\section*{Selected Presentations}

\begin{itemize}
	\item 2021. \emph{Land Invasions and Contemporary Slavery.}
	Southern Political Science Association,  Online, January.
	\item 2021. \emph{Conditional Cash Transfers and Child Labor: Insights from Brazil's Bolsa Familia.} Southern Political Science Association,  Online, January (with Taylor K. Chewning, Amanda Driscoll, and Marco A. Faganello).
	\item 2021. \emph{Land Invasions and Contemporary Slavery.} Latin American POLMETH, Online, November.
	\item 2020. \emph{Can Conditional Cash Transfers Interrupt the Cycle of Intergenerational Poverty? Lessons from a Large Administrative Dataset.} Latin American POLMETH, Online, November (with Amanda Driscoll).
	\item 2020. \emph{Political Regimes and Deaths in the Early Stages of the COVID-19 Pandemic.} Workshop "Rising Democracies, Interrupted,$"$  organized as a webinar by the Research Committee "Quality of Democracy$"$ of the International Political Science Association (IPSA), June (with Michael T. Dorsch, and Réka Branyiczki).
	\item 2019. \emph{When Does Clientelism Pay Off? Legislature Size and Welfare in Brazil.} LASA's International Congress, Boston, United States, May (with Umberto Mignozzetti). 
	\item 2018. \emph{When Does Clientelism Pay Off? Legislature Size and Welfare in Brazil.} IPSA World Congress of Political Science, Brisbane, Australia, July (with Umberto Mignozzetti).
	\item  2016. \emph{Politicians Matter: Legislature Size and Welfare with Evidence from Brazil}, and \emph{Bias and Discrimination in International Courts.} Uppsala University, Uppsala, Sweden, September-October (with Umberto Mignozzetti).
	\item 2014. \emph{Is the Dispute Settlement Body Impartial? Observational}, and \emph{Experimental Evidence from the World Trade Organization} and \emph{Do We Really Want More Politicians? The impact of the Size of Legislatures on Quality and Performance of Government.} Midwest Political Science Association, Chicago, Illinois, United States, April (with Umberto Mignozzetti).
	\item \emph{United We Stand Divided We Fall: Which are the Countries that Join into Coalitions More Often in the GATT/WTO Negotiations?.} IPSA World Congress, Montréal, Québec, Canada, July, 2014 (with Ivan F. Fernandes).
	\item 2014. \emph{A Roundtable on the 2014 Brazilian Elections.} Brown University, Providence, Rhode Island, United States, Sept. (with James N. Green, and Anani Dzidzienyo).
	\item 2012. \emph{On Politicians' Office-Seeking Behavior: Evidence from a Pair of Field Experiments in Brazil.} IPSA World Congress, part of the Federalism and Inequality in the Global South BIARI Alumni Initiative, funded by Mellon-LASA, Santander Universities, and Brown University, Madrid, Spain, July (with Feliciano D. S. Guimarães, Paolo Spada, and Holger L. Kern).
\end{itemize}

\section*{Grants and Awards}

\begin{itemize}
	\item \officialeuro300,000 --- \emph{European Transoceanic Encounters and Exchanges (ETEE).} Jean Monnet Network, Erasmus+ Programme of the EU, (Patrick Pasture (PI), and with researchers from University of Leuven, Belgium; University of California, Berkeley, United States; Kobe University, Japan; Nanyang Technological University, Singapore; UNESP, Brazil; Seoul National University, South Korea; University of Amsterdam, The Netherlands; and Jawaharlal Nehru University, India), 2019-2022.
	\item US\$14,000 --- \emph{How Political Competition Fuels Conditional Cash Transfer Programs: The Case of Brazil’s Bolsa Família Program.}  Florida State University Council for Research and Creativity Grant Program, 2019 (with Amanda Driscoll --- PI).
	\item US\$3,000 --- \textbf{Cepaluni, Gabriel} (PI). \emph{Experimental Studies on Social Norms.} FAPESP, São Paulo, Brazil, 2018-2019. (with Esteban Freidin).
	\item US\$10,000 --- \textbf{Cepaluni, Gabriel} (PI). \emph{Relationship between the Number of Legislators and Social Well-Being.} FAPESP, São Paulo, Brazil, 2017 - 2019.
	\item US\$2,000 --- \textbf{Cepaluni, Gabriel} (PI). \emph{Is the Dispute Settlement Body Impartial?: Observational and Experimental Evidence from the World Trade Organization.} CNPq, Brasília, Brazil, 2014-2017.
	\item US\$10,000 --- \textbf{Cepaluni, Gabriel} (PI). TOP USA Massachusetts Award (with Rebecca Weitz-Shapiro), 2013.
	\item US\$25,000 --- Federalism and Inequality in the Global South BIARI Alumni Initiative --- BIARI Alumni Seed Fund Award, Mellon-LASA, Santander Universities, and Brown University, 2011-2014 (Richard Snyder (PI) and Lorena Moscovich (CO-PI)).
	\item US\$30,000 --- Ph.D. scholarship, CNPq, Brasília, Brazil, 2005-2010.
	\item US\$7,500 --- MA scholarship, CAPES, Brasília, Brazil, 2003-2004.
\end{itemize}

\section*{Additional Training}

\begin{itemize}
	\item  2014. \emph{Introduction to Network Analysis using Pajek.}  IPSA Summer School, São Paulo, Brazil (taught by Prof. Vladimir Batagelj, University of Ljubljana).
	\item 2012. \emph{Survey Experiments in Political Science.}   Department of Political Science, University of São Paulo (DCP-USP), Brazil (taught by Prof. Matthew S. Winters, University of Illinois at Urbana Champaign).
	\item 2012. \emph{A Practical Introduction to Bayesian Statistical Modeling.}  IPSA Summer School, São Paulo, Brazil (taught by Prof. Simon Jackman, Stanford University).
	\item  2011. \emph{Multiple Regression Analysis.}  IPSA Summer School, São Paulo, Brazil (taught by Prof. Guy D. Whitten, Texas A$\&$M University).
	\item 2012. \emph{Multilevel Analyses in Comparative Politics.}   Department of Political Science, University of São Paulo (DCP-USP), Brazil (taught by Prof. Ernesto Calvo, University of Maryland).
	\item 2010. \emph{Development and Inequality in the Global South.}  Brown International Advanced Research Institutes, BIARI, United States
	\item  2010. \emph{Experimental Methods in Social Sciences.}  Getulio Vargas Foundation (FGV-SP), São Paulo, Brazil (taught by Prof. Scott W. Desposato, University of California, San Diego).
	\item 2010. \emph{Mathematical Concepts and Formal Modeling.}  IPSA Summer School, São Paulo, Brazil (Prof. Rebecca Morton, New York University).
\end{itemize}

\section*{Teaching}

\begin{itemize}
	\item Experimental Approach to Public Policy Evaluation (Graduate)
	\item Research Design (Undergraduate and Graduate)
	\item Foreign Policy and Regional Integration (Undergraduate)
	\item International Trade (Undergraduate)
	\item Introduction to International Affairs (Undergraduate)
	\item International Negotiations (Undergraduate)
	\item Foreign Policy in Comparative Perspective (Undergraduate)
\end{itemize}

\section*{Advising}

\textbf{Ph.D. Committees:}  Rodolpho T. Bernabel (USP, São Paulo) and  Armando Gallo Yahn Filho (UNICAMP, São Paulo).

\textbf{Master's Committees:} Jorge Luiz Pigini de Freitas (Chair, UNESP, São Paulo), 
Murilo Celli (Chair, UNESP, São Paulo), Beibei Sang (\emph{Programa San Tiago Dantas}, São Paulo), Gabriel Ciucci (\emph{Universidad Nacional del Sur}, Argentina), André Ziccardi Gomes Nogueira (USP, São Paulo), Douglas Henrique Novelli (UFPR, Paraná), Thiago Tâm Huynh Trung (USP, São Paulo), José Flávio de Castro (UNESP).

\section*{Skills}

\subsection*{Software}

\begin{itemize}
	\item R, Stata, \LaTeX{}.
\end{itemize}

\subsection*{Languages}

\begin{itemize}
	\item Portuguese (native), English (fluent), Spanish (fluent), Italian (basic), French (basic).
\end{itemize}

\section*{Service}

\begin{itemize}
	\item \textbf{Organizer:} \emph{Workshop: Political Economy of Development: Democratic Institutions and Poverty Alleviation.} Central European University, Budapest, Hungary, supported by IAS-CEU, SPP-CEU, and Global Challenges Fellowship, February 2020 (co-organizers: Amanda Driscoll and Cristina Corduneanu-Huci).
	\item \textbf{Reviewer (English journals):} American Political Science Review, Journal of Politics, Political Analysis, International Studies Quarterly, Plos One, Review of Developing Economics, Latin American Research Review, Latin American Politics and Society, Journal of Politics in Latin America,  Foreign Policy Analysis, Review of  International Studies, Political Geography, Frontiers in Public Health, Population Review,  Journal of Public Finance and Public Choice, Politics,  International Journal of Sociology, Globalizations, Cross-Cultural Research, Brazilian Political Science Review.
\end{itemize}

\section*{References}

\begin{itemize}
	
	\item Amanda Driscoll. Associate Professor, Department of Political Science, Florida State University. Email: \href{mailto:adriscoll@fsu.edu}{adriscoll@fsu.edu}.
	\item Matthew S. Winters. Professor, Department of Political Science, University of Illinois at Urbana-Champaign. Email: \href{mailto:mwinters@illinois.edu}{mwinters@illinois.edu}.
	\item Michael T. Dorsch. Associate Professor, Department of Public Policy, Central European University. Email: \href{mailto:DorschM@ceu.edu}{DorschM@ceu.edu}.
	\item Michael Touchton. Associate Professor, Department of Political Science, University of Miami. Email: \href{mailto:miketouchton@miami.edu}{miketouchton@miami.edu}.
\end{itemize}

	\bigskip


	\end{document}
